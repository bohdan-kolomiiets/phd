
%%%%%%%%%%%%%  Start of GENERATED TEXT  %%%%%%%%%%%%\n

\newpage
\setcounter{page}{1}
\setcounter{section}{0}
\setcounter{equation}{0}
\setcounter{figure}{0}
\setcounter{table}{0}
\setcounter{footnote}{0}
\renewcommand{\ehkol}{Kolomiiets B. Yu., Karplyuk Ye. S.}
\renewcommand{\ohkol}{Improving sEMG Gesture Classification with Transfer Learning: Accuracy, Stability, and Low-Data Adaptation}
\Vykh{radap20NN:FirstPage}{radap20NN:LastPage}
\begin{multicols}{2}
[
\UDC{}
\renewcommand{\ehkol}{Kolomiiets B. Yu., Karplyuk Ye. S.}
\renewcommand{\ohkol}{Improving sEMG Gesture Classification with Transfer Learning: Accuracy, Stability, and Low-Data Adaptation}
\Title{Improving sEMG Gesture Classification with Transfer Learning: Accuracy, Stability, and Low-Data Adaptation}\label{radap20NN:FirstPage}
\Authors{Kolomiiets B. Yu., Karplyuk Ye. S.}
\aff{National Technical University of Ukraine "Igor Sikorsky Kyiv Polytechnic Institute", Kyiv, Ukraine}
\Address{}{kolomiiets.bohdan.yuriyovych@gmail.com}
\AbsKeywords{
Surface electromyography (sEMG) enables intuitive and non-invasive human–machine interfacing, yet its practical deployment remains limited by inter-subject variability and the substantial effort required for subject-specific model calibration. In this work, we evaluate the effectiveness of transfer learning to address these challenges in the context of deep learning-based gesture classification, using a dataset that includes eleven gestures repeated eight times by 22 able-bodied subjects. We systematically compare three training approaches: (i) intra-subject training, (ii) inter-subject generalization, and (iii) transfer learning with and without resetting the convolutional neural networks (CNN) classifier’s fully connected (FC) output layer. A rigorous leave-one-out cross-validation scheme was applied across all subjects and repetitions to ensure robust evaluation.
Our results show that both transfer learning strategies outperform other two models in terms of classification accuracy, with the best performance achieved when fine-tuning with a reset FC layer (F1-score = 0.907, $\sigma = 0.074$). Statistical analysis using Wilcoxon signed-rank tests confirms that these improvements are significant, even when only a limited number of subject-specific repetitions are used for fine-tuning. Notably, the transfer learning approach being trained on four repetitions maintains almost the same performance compared to training from scratch on eight repetitions of one subject, halving training repetitions while preserving accuracy. This represents a substantial reduction in calibration effort without significantly sacrificing performance.
These findings support the use of pre-trained models for rapid subject adaptation in sEMG-controlled systems and validate fine-tuning with FC reset as an effective strategy for improving both model's performance and stability. The methodology and results presented in this work contribute to the development of practical, adaptive, and low-effort EMG-based interfaces for assistive and rehabilitative applications.
}
{Keywords}{
surface electromyography (sEMG), gesture recognition, transfer learning, deep learning, convolutional neural networks (CNN), inter-subject variability, model generalization, fine-tuning strategies, subject adaptation, myoelectric control, EMG-based interface, cross-validation, biomedical signal processing, rehabilitation technologies, low-effort calibration
 }
{\href{http://radap.kpi.ua/radiotechnique/article/view/20NN}{10.20535/RADAP.2025.\#\#.\pageref{radap20NN:FirstPage}-\pageref{radap20NN:LastPage}}}
]

%
%

\pdfbookmark[1]{Вступ}{intro}
\section*{Вступ}

 
\section{Назва}


\section{Назва}


\section{Приклади оформлення окремих елементів статті}

Формули з скрізним номером: 
\begin{equation}\label{radap1725eq1}
p_n(z(\xi)) = \frac{1}{\sqrt{2\pi} v} \exp  \left(-\frac{1}{2}\Bigg(\frac{z(\xi)}{v} \Bigg)^2 \right)
\end{equation}

\begin{equation}\label{radap1725eq2}
p_{cn}(z(\xi)) = \frac{1}{\sqrt{2\pi} v} \exp  \left(-\frac{1}{2}\Bigg(\frac{z(\xi)}{v}-q \Bigg)^2 \right) 
\end{equation}

Декілька рівнянь з позначкою одним номером:
\begin{equation*}
\begin{aligned}
{K_1}\left( {x,y} \right) = {C_1}\left( {x,y} \right) + {n_1}\left( {x,y} \right),\\	
{K_2}\left( {x,y} \right) = {C_2}\left( {x,y} \right) + {n_2}\left( {x,y} \right),
\end{aligned}
\end{equation*}

%\begin{equation}
%\begin{aligned}
%
%\end{aligned}
%\end{equation}

Довгі формули слід записувати у декілька стрічок як це приведено нижче:
\begin{multline}\label{radap1345eq2}
U_{i-1}= \sum\limits_{j=0}^{i-1}\left| r_j - \sum\limits_{h=0}^{g} x_{(j-h)} \nu_h \right|^2 =\\= \sum\limits_{j=0}^{i-2}\left| r_j - \sum\limits_{h=0}^{g} x_{(j-h)} \nu_h \right|^2 +w_{i-1} =\\= U_{i-2} + w_{i-1} 
\end{multline}

Дуже довгі формули, що важко розмістити в одній колонці можна розмістити на всю сторінку як це приведено нижче:

\end{multicols} % Закриваємо розмітку на дві колонки
\begin{multline}
\upsilon({{s}_{x}},{{s}_{y}},{{s}_{z}})\!=\!\frac{\left[{{p}_{\bot }}{{J}_{1}}\left(\sqrt{s_{x}^{2}+s_{y}^{2}}{{r}_{0}}\right){{J}_{0}}({{p}_{\bot }})-\sqrt{s_{x}^{2}+s_{y}^{2}}{{r}_{0}}{{J}_{0}}\left(\sqrt{s_{x}^{2}+s_{y}^{2}}{{r}_{0}}\right){{J}_{1}}({{p}_{\bot }})\right]}{(s_{x}^{2}+s_{y}^{2})r_{0}^{2}-p_{\bot }^{2}}\frac{\omega _{z}^{*}({{s}_{z}})}{\sqrt{s_{x}^{2}+s_{y}^{2}}};  
\end{multline}
\begin{multicols}{2} % Відкриваємо нову розмітку на дві колонки 

Формули без номеру: 
$$
M\left\{ {\ln \left( {\left. \Lambda  \right|\xi } \right)} \right\} = \frac{{{{\left( {{\rm M}\left\{ {z\left( \xi  \right)} \right\}} \right)}^2}}}{{2{v^2}}}.
$$	
 
   
  

\section{ }

\subsection{Рисунки}
 
\begin{Figure}\centering%{l}{\linewidth}
\includegraphics[width=\linewidth]{fig1}
\captionof{figure}{Графік залежності помилки визначення висот об'єктів від значення базису стереознімання}\label{radap1627fig1}
\end{Figure}
 
\begin{figure*}\centering
	%Figure 5 	
	\includegraphics[width=0.4\linewidth]{fig2a}
	~~~~~
	\includegraphics[width=0.4\linewidth]{fig2b}
	\begin{tabular}{p{0.49\linewidth}p{0.49\linewidth}}
		\centering (a) & \centering (b)  
	\end{tabular}	
	\captionof{figure}{Підпис до рисунку (a) $\Delta\delta_{DD}(r_n)$ (b)  $\theta = 1 $}\label{fig2}%
\end{figure*}

\subsection{Оформлення таблиць}

Таблиця в одній колонці
\begin{Table}
	\captionof{table}{Назва таблиці}
	\begin{tabularx}{\linewidth}{|l|c|c|c|c|X|}
		\hline   &   &  &  &  &  \\
		\rule{0pt}{10pt}   &   &	 &	  & &	 \\ 
		\hline 
		\rule{0pt}{10pt}  &  &   &	  &	  &  \\ 
		\hline 
		\rule{0pt}{10pt}  &   &	 &	  &	  &	 \\ 
		\hline 
		\rule{0pt}{10pt}  &   &   &	  &	  & \\ 
		\hline 
		\rule{0pt}{10pt}   &	  &  &   &	  &  \\ 
		\hline 
\end{tabularx} \label{radap1725tab1}
\end{Table}

Таблиця на дві колонки
\end{multicols}
\begin{Table}
\captionof{table}{Назва таблиці}
\begin{tabularx}{\linewidth}{|l|c|c|c|c|X|}
	\hline   &   &  &  &  &  \\
	\rule{0pt}{10pt}   &   &	 &	  & &	 \\ 
	\hline 
	\rule{0pt}{10pt}  &  &   &	  &	  &  \\ 
	\hline 
	\rule{0pt}{10pt}  &   &	 &	  &	  &	 \\ 
	\hline 
	\rule{0pt}{10pt}  &   &   &	  &	  & \\ 
	\hline 
	\rule{0pt}{10pt}   &	  &  &   &	  &  \\ 
	\hline 
\end{tabularx} \label{radap1725tab2}
\end{Table}
\begin{multicols}{2}




\pdfbookmark[1]{Висновки}{conc}
\section*{Висновки}



\pdfbookmark[1]{Перелік посилань}{lit}
\section*{Перелік посилань}

\href{http://radap.kpi.ua/radiotechnique/citing}{Правила оформлення посилань}

\pdfbookmark[1]{References}{translit}
\renewcommand{\refname}{References}

\begin{thebibliography}{99}\footnotesize 
	
	\bibitem{radap1725ref1} Omelianenko M., Romanenko T. (2020). E-plane Stepped-Impedance Bandpass Filter with Wide Stopband. \href{https://ieeexplore.ieee.org/abstract/document/9088888}{\textit{2020 IEEE 40th International Conference on Electronics and Nanotechnology (ELNANO)}}, pp. 838-841, doi: 10.1109/ELNANO50318.2020.9088888.
	
	\bibitem{radap1725ref2}	 
	
	\bibitem{radap1725ref3}	 
	
	\bibitem{radap1725ref4}	 
	
	\bibitem{radap1725ref5}	 
	
	\bibitem{radap1725ref6}	 
	
	\bibitem{radap1725ref7}	 
	
	\bibitem{radap1725ref8}	 
	
	\bibitem{radap1725ref9}	 
	
	\bibitem{radap1725ref10}
	
	\bibitem{radap1725ref11} 
	
	\bibitem{radap1725ref12}	 
	
	\bibitem{radap1725ref13}	 
	
	\bibitem{radap1725ref14}	 
	
	\bibitem{radap1725ref15}	 	 
	
\end{thebibliography}  

\TitleSecond{Improving sEMG Gesture Classification with Transfer Learning: Accuracy, Stability, and Low-Data Adaptation}
\Auth{Kolomiiets B. Yu., Karplyuk Ye. S.}
\AbsK{
Surface electromyography (sEMG) enables intuitive and non-invasive human–machine interfacing, yet its practical deployment remains limited by inter-subject variability and the substantial effort required for subject-specific model calibration. In this work, we evaluate the effectiveness of transfer learning to address these challenges in the context of deep learning-based gesture classification, using a dataset that includes eleven gestures repeated eight times by 22 able-bodied subjects. We systematically compare three training approaches: (i) intra-subject training, (ii) inter-subject generalization, and (iii) transfer learning with and without resetting the convolutional neural networks (CNN) classifier’s fully connected (FC) output layer. A rigorous leave-one-out cross-validation scheme was applied across all subjects and repetitions to ensure robust evaluation.
Our results show that both transfer learning strategies outperform other two models in terms of classification accuracy, with the best performance achieved when fine-tuning with a reset FC layer (F1-score = 0.907, $\sigma = 0.074$). Statistical analysis using Wilcoxon signed-rank tests confirms that these improvements are significant, even when only a limited number of subject-specific repetitions are used for fine-tuning. Notably, the transfer learning approach being trained on four repetitions maintains almost the same performance compared to training from scratch on eight repetitions of one subject, halving training repetitions while preserving accuracy. This represents a substantial reduction in calibration effort without significantly sacrificing performance.
These findings support the use of pre-trained models for rapid subject adaptation in sEMG-controlled systems and validate fine-tuning with FC reset as an effective strategy for improving both model's performance and stability. The methodology and results presented in this work contribute to the development of practical, adaptive, and low-effort EMG-based interfaces for assistive and rehabilitative applications.
}
{
surface electromyography (sEMG), gesture recognition, transfer learning, deep learning, convolutional neural networks (CNN), inter-subject variability, model generalization, fine-tuning strategies, subject adaptation, myoelectric control, EMG-based interface, cross-validation, biomedical signal processing, rehabilitation technologies, low-effort calibration
}

\label{radap20NN:LastPage}
\end{multicols}
\newpage
%%%%%%%%%%%%%  END of GENERATED TEXT  %%%%%%%%%%%%\n