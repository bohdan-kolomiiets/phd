
%%%%%%%%%%%%%  Start of GENERATED TEXT  %%%%%%%%%%%%\n

\newpage
\setcounter{page}{1}
\setcounter{section}{0}
\setcounter{equation}{0}
\setcounter{figure}{0}
\setcounter{table}{0}
\setcounter{footnote}{0}
\renewcommand{\ehkol}{Kolomiiets B. Yu., Karpliuk Ye. S.}
\renewcommand{\ohkol}{Improving EMG Signal Classification with Transfer Learning Under Low-Data and Cross-Subject Conditions}
\Vykh{radap20NN:FirstPage}{radap20NN:LastPage}
\begin{multicols}{2}
[
\UDC{}
\renewcommand{\ehkol}{Kolomiiets B. Yu., Karpliuk Ye. S.}
\renewcommand{\ohkol}{Improving EMG Signal Classification with Transfer Learning Under Low-Data and Cross-Subject Conditions}
\Title{Improving EMG Signal Classification with Transfer Learning Under Low-Data and Cross-Subject Conditions}\label{radap20NN:FirstPage}
\Authors{Kolomiiets B. Yu., Karpliuk Ye. S.}
\aff{National Technical University of Ukraine "Igor Sikorsky Kyiv Polytechnic Institute", Kyiv, Ukraine}
\Address{}{kolomiiets.bohdan.yuriyovych@gmail.com}
\AbsKeywords{
Surface electromyography (sEMG) is a non-invasive method used for human–machine interaction, but its practical use is often limited by large differences between individuals and the effort needed to train models for each new user. This study explores whether transfer learning can help solve these problems when using deep learning to classify hand and wrist gestures. The experiments were conducted on a dataset that includes eleven gestures, each repeated eight times by 22 healthy participants. Three training approaches were tested: (i) training and testing on the same subject (intra-subject), (ii) training on some subjects and testing on a new one (inter-subject), and (iii) transfer learning with and without resetting the fully connected (FC) output layer of the convolutional neural network (CNN). All models were evaluated using a leave-one-out cross-validation strategy across both subjects and repetitions.
Results show that both transfer learning methods performed better than the other two approaches in terms of classification accuracy. The best result was achieved when the FC layer was reset before fine-tuning (F1-score = 0.907, $\sigma = 0.074$). Wilcoxon signed-rank statistical tests confirmed that these improvements were statistically significant, even when only a few repetitions were used for fine-tuning. In fact, using transfer learning with just four repetitions instead of eight gave similar accuracy to training from scratch on eight repetitions.
These findings suggest that fine-tuning pre-trained models can significantly lower the effort required to adapt EMG-based systems for new users, offering a practical and effective approach for developing practical interfaces suited to assistive and rehabilitation applications.
}
{Keywords}{
surface electromyography (sEMG), gesture recognition, transfer learning, deep learning, convolutional neural networks (CNN), inter-subject variability, model generalization, fine-tuning strategies, subject adaptation, myoelectric control, EMG-based interface, cross-validation, biomedical signal processing, rehabilitation technologies, low-effort calibration
 }
{\href{http://radap.kpi.ua/radiotechnique/article/view/20NN}{10.20535/RADAP.2025.\#\#.\pageref{radap20NN:FirstPage}-\pageref{radap20NN:LastPage}}}
]

%
%

\pdfbookmark[1]{Вступ}{intro}
\section*{Вступ}

 
\section{Назва}


\section{Назва}


\section{Приклади оформлення окремих елементів статті}

Формули з скрізним номером: 
\begin{equation}\label{radap1725eq1}
p_n(z(\xi)) = \frac{1}{\sqrt{2\pi} v} \exp  \left(-\frac{1}{2}\Bigg(\frac{z(\xi)}{v} \Bigg)^2 \right)
\end{equation}

\begin{equation}\label{radap1725eq2}
p_{cn}(z(\xi)) = \frac{1}{\sqrt{2\pi} v} \exp  \left(-\frac{1}{2}\Bigg(\frac{z(\xi)}{v}-q \Bigg)^2 \right) 
\end{equation}

Декілька рівнянь з позначкою одним номером:
\begin{equation*}
\begin{aligned}
{K_1}\left( {x,y} \right) = {C_1}\left( {x,y} \right) + {n_1}\left( {x,y} \right),\\	
{K_2}\left( {x,y} \right) = {C_2}\left( {x,y} \right) + {n_2}\left( {x,y} \right),
\end{aligned}
\end{equation*}

%\begin{equation}
%\begin{aligned}
%
%\end{aligned}
%\end{equation}

Довгі формули слід записувати у декілька стрічок як це приведено нижче:
\begin{multline}\label{radap1345eq2}
U_{i-1}= \sum\limits_{j=0}^{i-1}\left| r_j - \sum\limits_{h=0}^{g} x_{(j-h)} \nu_h \right|^2 =\\= \sum\limits_{j=0}^{i-2}\left| r_j - \sum\limits_{h=0}^{g} x_{(j-h)} \nu_h \right|^2 +w_{i-1} =\\= U_{i-2} + w_{i-1} 
\end{multline}

Дуже довгі формули, що важко розмістити в одній колонці можна розмістити на всю сторінку як це приведено нижче:

\end{multicols} % Закриваємо розмітку на дві колонки
\begin{multline}
\upsilon({{s}_{x}},{{s}_{y}},{{s}_{z}})\!=\!\frac{\left[{{p}_{\bot }}{{J}_{1}}\left(\sqrt{s_{x}^{2}+s_{y}^{2}}{{r}_{0}}\right){{J}_{0}}({{p}_{\bot }})-\sqrt{s_{x}^{2}+s_{y}^{2}}{{r}_{0}}{{J}_{0}}\left(\sqrt{s_{x}^{2}+s_{y}^{2}}{{r}_{0}}\right){{J}_{1}}({{p}_{\bot }})\right]}{(s_{x}^{2}+s_{y}^{2})r_{0}^{2}-p_{\bot }^{2}}\frac{\omega _{z}^{*}({{s}_{z}})}{\sqrt{s_{x}^{2}+s_{y}^{2}}};  
\end{multline}
\begin{multicols}{2} % Відкриваємо нову розмітку на дві колонки 

Формули без номеру: 
$$
M\left\{ {\ln \left( {\left. \Lambda  \right|\xi } \right)} \right\} = \frac{{{{\left( {{\rm M}\left\{ {z\left( \xi  \right)} \right\}} \right)}^2}}}{{2{v^2}}}.
$$	
 
   
  

\section{ }

\subsection{Рисунки}
 
\begin{Figure}\centering%{l}{\linewidth}
\includegraphics[width=\linewidth]{fig1}
\captionof{figure}{Графік залежності помилки визначення висот об'єктів від значення базису стереознімання}\label{radap1627fig1}
\end{Figure}
 
\begin{figure*}\centering
	%Figure 5 	
	\includegraphics[width=0.4\linewidth]{fig2a}
	~~~~~
	\includegraphics[width=0.4\linewidth]{fig2b}
	\begin{tabular}{p{0.49\linewidth}p{0.49\linewidth}}
		\centering (a) & \centering (b)  
	\end{tabular}	
	\captionof{figure}{Підпис до рисунку (a) $\Delta\delta_{DD}(r_n)$ (b)  $\theta = 1 $}\label{fig2}%
\end{figure*}

\subsection{Оформлення таблиць}

Таблиця в одній колонці
\begin{Table}
	\captionof{table}{Назва таблиці}
	\begin{tabularx}{\linewidth}{|l|c|c|c|c|X|}
		\hline   &   &  &  &  &  \\
		\rule{0pt}{10pt}   &   &	 &	  & &	 \\ 
		\hline 
		\rule{0pt}{10pt}  &  &   &	  &	  &  \\ 
		\hline 
		\rule{0pt}{10pt}  &   &	 &	  &	  &	 \\ 
		\hline 
		\rule{0pt}{10pt}  &   &   &	  &	  & \\ 
		\hline 
		\rule{0pt}{10pt}   &	  &  &   &	  &  \\ 
		\hline 
\end{tabularx} \label{radap1725tab1}
\end{Table}

Таблиця на дві колонки
\end{multicols}
\begin{Table}
\captionof{table}{Назва таблиці}
\begin{tabularx}{\linewidth}{|l|c|c|c|c|X|}
	\hline   &   &  &  &  &  \\
	\rule{0pt}{10pt}   &   &	 &	  & &	 \\ 
	\hline 
	\rule{0pt}{10pt}  &  &   &	  &	  &  \\ 
	\hline 
	\rule{0pt}{10pt}  &   &	 &	  &	  &	 \\ 
	\hline 
	\rule{0pt}{10pt}  &   &   &	  &	  & \\ 
	\hline 
	\rule{0pt}{10pt}   &	  &  &   &	  &  \\ 
	\hline 
\end{tabularx} \label{radap1725tab2}
\end{Table}
\begin{multicols}{2}




\pdfbookmark[1]{Висновки}{conc}
\section*{Висновки}



\pdfbookmark[1]{Перелік посилань}{lit}
\section*{Перелік посилань}

\href{http://radap.kpi.ua/radiotechnique/citing}{Правила оформлення посилань}

\pdfbookmark[1]{References}{translit}
\renewcommand{\refname}{References}

\begin{thebibliography}{99}\footnotesize 
	
	\bibitem{radap1725ref1} Omelianenko M., Romanenko T. (2020). E-plane Stepped-Impedance Bandpass Filter with Wide Stopband. \href{https://ieeexplore.ieee.org/abstract/document/9088888}{\textit{2020 IEEE 40th International Conference on Electronics and Nanotechnology (ELNANO)}}, pp. 838-841, doi: 10.1109/ELNANO50318.2020.9088888.
	
	\bibitem{radap1725ref2}	 
	
	\bibitem{radap1725ref3}	 
	
	\bibitem{radap1725ref4}	 
	
	\bibitem{radap1725ref5}	 
	
	\bibitem{radap1725ref6}	 
	
	\bibitem{radap1725ref7}	 
	
	\bibitem{radap1725ref8}	 
	
	\bibitem{radap1725ref9}	 
	
	\bibitem{radap1725ref10}
	
	\bibitem{radap1725ref11} 
	
	\bibitem{radap1725ref12}	 
	
	\bibitem{radap1725ref13}	 
	
	\bibitem{radap1725ref14}	 
	
	\bibitem{radap1725ref15}	 	 
	
\end{thebibliography}  

\TitleSecond{Improving EMG Signal Classification with Transfer Learning Under Low-Data and Cross-Subject Conditions}
\Auth{Kolomiiets B. Yu., Karpliuk Ye. S.}
\AbsK{
Surface electromyography (sEMG) is a non-invasive method used for human–machine interaction, but its practical use is often limited by large differences between individuals and the effort needed to train models for each new user. This study explores whether transfer learning can help solve these problems when using deep learning to classify hand and wrist gestures. The experiments were conducted on a dataset that includes eleven gestures, each repeated eight times by 22 healthy participants. Three training approaches were tested: (i) training and testing on the same subject (intra-subject), (ii) training on some subjects and testing on a new one (inter-subject), and (iii) transfer learning with and without resetting the fully connected (FC) output layer of the convolutional neural network (CNN). All models were evaluated using a leave-one-out cross-validation strategy across both subjects and repetitions.
Results show that both transfer learning methods performed better than the other two approaches in terms of classification accuracy. The best result was achieved when the FC layer was reset before fine-tuning (F1-score = 0.907, $\sigma = 0.074$). Wilcoxon signed-rank statistical tests confirmed that these improvements were statistically significant, even when only a few repetitions were used for fine-tuning. In fact, using transfer learning with just four repetitions instead of eight gave similar accuracy to training from scratch on eight repetitions.
These findings suggest that fine-tuning pre-trained models can significantly lower the effort required to adapt EMG-based systems for new users, offering a practical and effective approach for developing practical interfaces suited to assistive and rehabilitation applications.
}
{
surface electromyography (sEMG), gesture recognition, transfer learning, deep learning, convolutional neural networks (CNN), inter-subject variability, model generalization, fine-tuning strategies, subject adaptation, myoelectric control, EMG-based interface, cross-validation, biomedical signal processing, rehabilitation technologies, low-effort calibration
}

\label{radap20NN:LastPage}
\end{multicols}
\newpage
%%%%%%%%%%%%%  END of GENERATED TEXT  %%%%%%%%%%%%\n